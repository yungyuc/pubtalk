\documentclass[dvips,xcolor=pst,14pt]{beamer}
%\documentclass[dvips,handout,xcolor=pst]{beamer}
%\documentclass[letterpaper]{article}
%\usepackage{beamerarticle}
%\newcommand{\newblock}{} % a hack for bibtex.
\usepackage{color}
\usepackage{pstricks}
%\usepackage{hyperref}
%\usepackage[authoryear]{natbib}
\usepackage{graphicx}
\usepackage{multimedia}
\usepackage{pgfpages}
\usepackage{arydshln}
\usepackage{commath}
\usepackage{vector}
%\usepackage{cancel}
\usepackage{ulem}
\usepackage{listings}
%\usepackage{times}

\graphicspath{{pysci_eps/}}

\newcommand{\defeq}{\ensuremath{\buildrel {\text{def}}\over{=}}} 

\usetheme{CambridgeUS}
%\usefonttheme[stillsansseriftext]{serif}
\usefonttheme[onlymath]{serif}

%\setbeameroption{show notes}
%\setbeameroption{show only notes}
%\pgfpagesuselayout{2 on 1}[letterpaper,border shrink=0.2in]

\lstset{basicstyle=\footnotesize\ttfamily,%
keywordstyle=\color{blue},%
stringstyle=\color{magenta},%
commentstyle=\color{green},%
breaklines=true}
%\lstset{framextopmargin=50pt,frame=bottomline}

\title[Cython for HPC]{Introduction to Cython: Bridging Python to Compiled Code
for High-Performance Computing}
%
\author[\href{http://solvcon.net/yyc/}{Yung-Yu Chen}]%
{\href{http://solvcon.net/yyc/}{Yung-Yu Chen} \\ {\scriptsize
\url{yyc@solvcon.net}}}
%
\institute[PyHUG]{Python Hsinchu User Group}
%
\date[2012/12/3]{December 3, 2012}

\begin{document}

\begin{frame}
\titlepage
\end{frame}

% \begin{frame}{
% %
% Our Topics Today
% %
% }
% \tableofcontents
% \end{frame}

\section{
%%%
Cython Is a Code Generator
%%%
}

\subsection{
%%
Introduction
%%
}

\begin{frame}{
%
What Is Cython?
%
}
\begin{itemize}
\item Cython generates C code from Python.
\item The generated code contains API calls to CPython runtime (VM).
\begin{itemize}
  \item Can interface to both the existing \alert{C} and \alert{C++} code.
\end{itemize}
\item Popular use cases:
\begin{itemize} \normalsize
  \item Convert Python code into binary for protection (and a little bit
  speed-up).
  \item Highly-tuned, high-performance C code.
  \item Glue Python with C.
\end{itemize}
\item \url{http://www.cython.org/}.
\end{itemize}
\end{frame}

\begin{frame}[fragile]{
%
Installation
%
}
\begin{itemize}
\item The easiest way is to use your OS' package manager, e.g., under
Debian/Ubuntu:
\begin{lstlisting}[language=bash]
> apt-get install cython
\end{lstlisting}
\item Use setuptools (\verb+easy_install+):
\begin{lstlisting}[language=bash]
> easy_install --upgrade cython
\end{lstlisting}
\item I am focusing on the most recent version of Cython:
\begin{lstlisting}[language=bash]
> cython --version
Cython version 0.17.2
\end{lstlisting}
\item If you don't want to mess up your system Python, virtualenv can help you:
\begin{lstlisting}[basicstyle=\scriptsize\ttfamily,language=bash]
> virtualenv --system-site-package --setuptools cpython
\end{lstlisting}
\end{itemize}
\end{frame}

\subsection{
%%
Basic Usage
%%
}

\begin{frame}[fragile]{
%
Build Your First Cython Module
%
}
\begin{itemize}
\item Let's say we have a Python file \verb+example1.py+:
\lstinputlisting[basicstyle=\scriptsize\ttfamily,language=Python]{example1.py}
\item It can be compiled by Cython (and gcc):
\begin{lstlisting}[basicstyle=\scriptsize\ttfamily,language=bash]
> cp example1.py /tmp/example1c.py
> cython /tmp/example1c.py -o /tmp/example1c.c
> gcc `python-config --includes` -c /tmp/example1c.c -fPIC -o /tmp/example1c.o
> gcc `python-config --libs` /tmp/example1c.o -shared -o example1c.so
\end{lstlisting}
\item The results are identical:
\begin{lstlisting}[basicstyle=\scriptsize\ttfamily,language=bash]
> python -c 'import example1; example1.action()'
> python -c 'import example1c; example1c.action()'
\end{lstlisting}
\end{itemize}
\end{frame}

\begin{frame}[fragile]{
%
Source Code Are Now Hidden
%
}
\begin{itemize}
\item Python code becomes binary for API calls:
\begin{lstlisting}[language=bash]
> file example1.py example1c.so 
example1.py:  ASCII text
example1c.so: ELF 64-bit LSB shared object, ...
\end{lstlisting}
\item But the runtime speed is not improved much:
\begin{lstlisting}[language=bash]
> python -m timeit -s 'from example1 import action' 'action()'
10 loops, best of 3: 49 msec per loop
> python -m timeit -s 'from example1c import action' 'action()'
10 loops, best of 3: 42.6 msec per loop
\end{lstlisting}
\item It's still run on interpreter, so the speed-up (42.6 msec vs 49 msec)
isn't impressive.
\end{itemize}
\end{frame}

\begin{frame}[fragile]{
%
Cython is a Language
%
}
\begin{itemize}
\item Cython extends Python.  It is a \alert{superset} of Python.
\begin{itemize}
\item When compiling {\color{red}\verb+.py+} files, of course only Python
syntax can be used.
\end{itemize}
\item If Cython-specific syntax is used, the source code can no longer be run
by a vanilla Python VM.
\begin{itemize}
\item It should be saved as a {\color{red}\verb+.pyx+} file and compiled by
Cython compiler to binary.
\end{itemize}
\item The Cython add-on helps to generate faster code and adapt to various
libraries.
\end{itemize}
\end{frame}

\section{
%%%
Use Cython for High-Performance Code
%%%
}

\subsection{
%%
Type Information
%%
}

\begin{frame}[fragile]{
%
Faster Code? Type Information
%
}
\begin{itemize}
\item Just prefix ``{\color{red}\verb+cdef int+}'' to our counter:
\begin{lstlisting}[basicstyle=\scriptsize\ttfamily,language=Python,escapechar=!]
def action():
    !\colorbox{red}{cdef int}! it = 0
    while it < 1000000:
        it += 1
    print it
\end{lstlisting}
and compile
\begin{lstlisting}[basicstyle=\scriptsize\ttfamily,language=bash]
> cython example1d.pyx -o /tmp/example1d.c
> gcc `python-config --includes` -c /tmp/example1d.c -fPIC -o /tmp/example1d.o
> gcc `python-config --libs` /tmp/example1d.o -shared -o example1d.so
\end{lstlisting}
\item Then 16 times faster! (3.01 msec vs 49 msec)
\begin{lstlisting}[language=bash]
> python -m timeit -s 'from example1d import action' 'action()'
100 loops, best of 3: 3.01 msec per loop
\end{lstlisting}
\end{itemize}
\end{frame}

\subsection{
%%
NumPy and Array Indexing
%%
}

\begin{frame}[fragile]{
%
Cython Supports NumPy
%
}
\begin{itemize}
\item You can use NumPy arrays ({\color{red}\verb+example2d.pyx+}):
\lstinputlisting[basicstyle=\tiny\ttfamily,language=Python]{example2d.pyx}
\end{itemize}
\end{frame}

\begin{frame}[fragile]{
%
Add Array Dimensions
%
}
\begin{itemize}
\item Let Cython use direct indexing ({\color{red}\verb+example2a.pyx+}):
\begin{lstlisting}[basicstyle=\tiny\ttfamily,language=Python,escapechar=!]
import numpy as np
cimport numpy as cnp
def action():
    !\colorbox{red}{cdef cnp.ndarray[cnp.double\_t, ndim=2]}! arr0 = np.empty(
        [1000,1000], dtype='float64')
    arr0.fill(0)
    !\colorbox{red}{cdef cnp.ndarray[cnp.double\_t, ndim=2]}! arr1 = np.empty(
        [1000,1000], dtype='float64')
    arr1.fill(1)
    cdef int it = 1
    cdef int jt
    while it < 999:
        jt = 1
        while jt < 999:
            arr0[it, jt] += arr1[it-1, jt-1]
            arr0[it, jt] += arr1[it-1, jt  ]
            arr0[it, jt] += arr1[it-1, jt+1]
            arr0[it, jt] += arr1[it  , jt+1]
            arr0[it, jt] += arr1[it+1, jt+1]
            arr0[it, jt] += arr1[it+1, jt  ]
            arr0[it, jt] += arr1[it+1, jt-1]
            arr0[it, jt] += arr1[it  , jt-1]
            jt += 1
        it += 1
    assert 7968032 == arr0.sum()
\end{lstlisting}
\end{itemize}
\end{frame}

\begin{frame}[fragile]{
%
Direct Indexing Is Way Faster
%
}
\begin{itemize}
\item Use NumPy API for indexing: 5.74 sec.
\begin{lstlisting}[language=bash]
> python -m timeit -s 'from example2d import action' 'action()'
10 loops, best of 3: 5.74 sec per loop
\end{lstlisting}
\item Direct indexing: 139 msec. 
\begin{lstlisting}[language=bash]
> python -m timeit -s 'from example2a import action' 'action()'
10 loops, best of 3: 139 msec per loop
\end{lstlisting}
\item Direct indexing makes the code run \alert{41} times faster.
\end{itemize}
\end{frame}

\begin{frame}[fragile]{
%
Turn off Bound-Checking
%
}
\begin{itemize}
\item Get a little speed-up ({\color{red}\verb+example2b.pyx+}): 110 msec.
\begin{lstlisting}[language=Python,escapechar=!]
import numpy as np
cimport numpy as cnp
!\colorbox{red}{cimport cython}!
!\colorbox{red}{@cython.boundscheck(False)}!
def action():
    cdef cnp.ndarray[cnp.double_t, ndim=2] arr0 = np.empty(
        [1000,1000], dtype='float64')
    arr0.fill(0)
    cdef cnp.ndarray[cnp.double_t, ndim=2] arr1 = np.empty(
        [1000,1000], dtype='float64')
    arr1.fill(1)
    ...
\end{lstlisting}
\end{itemize}
\end{frame}

\section{
%%%
Bridge C and Python
%%%
}

\begin{frame}[fragile]{
%
We Just Want to Call C
%
}
\begin{itemize}
\item Let's say we have a C function ({\color{red}\verb+example2.c+}):
\lstinputlisting[language=C]{example2.c}
\end{itemize}
\end{frame}

\begin{frame}[fragile]{
%
Cython Let You Call C
%
}
\begin{itemize}
\item {\color{red}\verb+example2c.pyx+}:
\lstinputlisting[language=Python]{example2c.pyx}
\end{itemize}
\end{frame}

\begin{frame}[fragile]{
%
Side Effect: Even Faster
%
}
\begin{itemize}
\item The C version is 51.4 msec, the fastest.
\begin{lstlisting}[language=bash]
> python -m timeit -s 'from example2c import action' 'action()'
10 loops, best of 3: 51.4 msec per loop
\end{lstlisting}
\item Compare to the original version, 5.74 sec / 51.4 msec = 106 times faster.
\item Compare to the pure-Cython version, 110 msec / 51.4 msec = 2 times
faster. 
\item \alert{C is the king}.
\begin{itemize}
  \item The rule of thumb: Python is \alert{two orders of magnitude slower}
  than C.
  \item But after we replace the Python hotspot with C, it's OK.
\end{itemize}
\end{itemize}
\end{frame}

\begin{frame}[fragile]{
%
Useful Syntax for Wrapping
%
}
\begin{itemize}
\item \verb+cdef extern from "header.h":+
\begin{itemize}
  \item Allows Cython to check the declaration from outside headers.
\end{itemize}
\item \verb+cdef public:+
\begin{itemize}
  \item Allows Cython to generates C header files for inclusion in other C
  code.
\end{itemize}
\end{itemize}
\end{frame}

\section{
%%%
Distribution with Cython
%%%
}

\subsection{
%%
pyximport
%%
}

\begin{frame}[fragile]{
%
Build Cython Code on the Fly
%
}
\begin{itemize}
\item For simple code, Cython can compile it on the fly by using
``{\color{red}\verb+pyximport+}'':
\begin{lstlisting}[language=Python]
import pyximport
pyximport.install()
\end{lstlisting}
\item Then standard \verb+import+ gets the \verb+pyx+ file compiled:
\begin{lstlisting}[language=Python]
import example2b
\end{lstlisting}
\item After that the code is ready for use:
\begin{lstlisting}[language=Python]
example2b.action()
\end{lstlisting}
\end{itemize}
\end{frame}

\subsection{
%%
Distutils with Cython
%%
}

\begin{frame}[fragile]{
%
Use Cython's Distutils Helper
%
}
\begin{itemize}
\item Prepare a \verb+setup.py+ file:
\lstinputlisting[language=Python]{setup.py}
\end{itemize}
\end{frame}

\begin{frame}[fragile]{
%
Build and Run the Extension
%
}
\begin{itemize}
\item Run the \verb+setup.py+:
\begin{lstlisting}[language=bash]
> python setup.py build_ext --inplace
\end{lstlisting}
\item We then can use the built module:
\begin{lstlisting}[language=bash]
> python -c 'from example2s import action; action()'
\end{lstlisting}
\end{itemize}
\end{frame}

\subsection{
%%
Standard Distutils
%%
}

\begin{frame}[fragile]{
%
Distribute Compiled C Files
%
}
\begin{itemize}
\item Prepare the C file to be distributed:
\begin{lstlisting}[language=bash]
> cython example2p.pyx -o example2p.c
\end{lstlisting}
\item Prepare another \verb+setup2.py+ file:
\lstinputlisting[language=Python]{setup2.py}
\end{itemize}
\end{frame}

\begin{frame}[fragile]{
%
Build and Run the Extension with Cython
%
}
\begin{itemize}
\item Run the \verb+setup2.py+:
\begin{lstlisting}[language=bash]
> python setup2.py build_ext --inplace
\end{lstlisting}
\alert{But no Cython is needed at building.}
\item We then can use the built module:
\begin{lstlisting}[language=bash]
> python -c 'from example2p import action; action()'
\end{lstlisting}
\end{itemize}
\end{frame}

\section*{
%%%
Conclusions
%%%
}

\begin{frame}{
%
What Cython Is Good At
%
}
\begin{itemize}
\item Hide the Python source code by turning it into C.
\begin{itemize}
  \item Even byte code disappears.  Only binary remains.
\end{itemize}
\item By supplying type information, provide an order of magnitude of speed-up.
\item When working with NumPy arrays, Cython supports direct indexing that
boosts the performance for tens of times.
\item Wrap self-written or 3rd-party libraries with great ease.
\end{itemize}
\end{frame}

\begin{frame}{
%
What You Shouldn't Do with Cython
%
}
\begin{itemize}
\item Write a complex class hierarchy and expect it to run very fast.
\begin{itemize}
  \item If you need high-performance code based on a complex class hierarchy,
  you should use C++ and wrap it around with Boost.Python, Cython, or something
  else.
\end{itemize}
\end{itemize}
\end{frame}

\begin{frame}{
%
Extended Resources
%
}
\begin{itemize}
\item \href{http://www.cython.org/}{Cython}
\item \href{http://www.boost.org/doc/libs/1_52_0/libs/python/doc/}%
{Boost.Python}
\item \href{http://www.swig.org/}{SWIG}
\item \href{http://docs.python.org/2/distutils/index.html}%
{Distributing Python Modules}
\item \href{http://docs.python.org/2/c-api/index.html}%
{Python C/API Reference Manual}
\end{itemize}
\end{frame}

\end{document}

% vim: set spell:
