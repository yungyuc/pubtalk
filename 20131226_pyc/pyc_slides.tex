\documentclass[dvips,xcolor=pst,14pt]{beamer}
%\documentclass[dvips,handout,xcolor=pst]{beamer}
%\documentclass[letterpaper]{article}
%\usepackage{beamerarticle}
%\newcommand{\newblock}{} % a hack for bibtex.
\usepackage{color}
\usepackage{pstricks}
%\usepackage{hyperref}
%\usepackage[authoryear]{natbib}
\usepackage{graphicx}
\usepackage{multimedia}
\usepackage{pgfpages}
\usepackage{arydshln}
\usepackage{commath}
\usepackage{vector}
%\usepackage{cancel}
\usepackage{ulem}
\usepackage{listings}
%\usepackage{times}

\graphicspath{{pyc_eps/}}

\newcommand{\defeq}{\ensuremath{\buildrel {\text{def}}\over{=}}} 

\usetheme{CambridgeUS}
%\usefonttheme[stillsansseriftext]{serif}
\usefonttheme[onlymath]{serif}

%\setbeameroption{show notes}
%\setbeameroption{show only notes}
%\pgfpagesuselayout{2 on 1}[letterpaper,border shrink=0.2in]

\lstset{basicstyle=\footnotesize\ttfamily,%
keywordstyle=\color{blue},%
stringstyle=\color{magenta},%
commentstyle=\color{green},%
breaklines=true}
%\lstset{framextopmargin=50pt,frame=bottomline}

\title[C/C++/Python]{C/C++ for Python's Use: You Wrap It}
%
\author[\href{http://solvcon.net/yyc/}{Yung-Yu Chen}]%
{\href{http://solvcon.net/yyc/}{Yung-Yu Chen} \\ {\scriptsize
\url{yyc@solvcon.net}}}
%
\institute[THF]{The Happy Folk}
%
\date[2013/12/26]{December 26, 2013}

\lstset{
xleftmargin=2em,
}

\begin{document}

\begin{frame}
\titlepage
\end{frame}

% \begin{frame}{
% %
% Our Topics Today
% %
% }
% \tableofcontents
% \end{frame}

\section{
%%%
Why Wrap
%%%
}

\begin{frame}{
%
Python is Slow, Way Too Slow
%
}
\begin{itemize}
\item The rule of thumb: Python is \alert{two orders of magnitude slower} than
C.
\begin{itemize}
  \item 10 milliseconds in C: 1 second in Python.
  \item 8 hours in C: a month in Python.
\end{itemize}
\item It's not OK to spend a month on an over-night job.
\begin{itemize}
  \item But we just can't lose the productivity by Python.
  \item So we want to connect Python with C (and/or C++).
  \item Another case: Python is the thin wrapper of the underlying system.
\end{itemize}
\end{itemize}
\end{frame}

\begin{frame}{
%
Ways to Wrapper
%
}
\begin{itemize}
\item \href{http://cython.org/}{\alert{Cython}} for C and
\href{http://www.boost.org/doc/libs/1_55_0/libs/python/doc/}%
{\alert{Boost.Python}} for C++.
\item There are other options:
\begin{itemize}
  \item \href{http://docs.python.org/2/library/ctypes.html}{ctypes} and/or
  \href{https://bitbucket.org/cffi/cffi}{cffi}.
  \item \href{http://www.swig.org/}{SWIG}.
  \item \href{Py++}{http://www.ohloh.net/p/pygccxml}.
  \item ... and many that we don't talk about.
\end{itemize}
\end{itemize}
\end{frame}

\begin{frame}{
%
In Conclusion ...
%
}
\begin{itemize}
\item If you don't need to deal with C++.  Congratulations.  You can just use
Cython without worrying anything.
\item If unfortunately you are working with C++.  Go Boost.Python.
\item {\small \url{http://docs.python.org/2/c-api/index.html}}
\begin{itemize}
  \item That is, use Python C API to make the low-level code available/callable
  in the Python interpreter.
  \item Nothing stops you from wrapping by your bare hands with the C API.
  \item But it's just an unpleasant approach (and error-prone).
\end{itemize}
\end{itemize}
\end{frame}

\section{
%%%
Boost.Python
%%%
}

\begin{frame}{
%
Let's Start with the Bad Boy
%
}
\begin{itemize}
\item C++ and Python both rely on their own OO systems.
\begin{itemize}
  \item Both systems are powerful but substantially different.
\end{itemize}
\item Boost.Python maps everything between C++ and Python.
\begin{itemize}
  \item Functions, classes, exceptions, containers, iterators, etc.
  \item It's \textit{like} writing Python extension with C++.
\end{itemize}
\end{itemize}
\end{frame}

\begin{frame}[fragile]{
%
Official Boost.Python Hello World
%
}
\begin{lstlisting}[basicstyle=\scriptsize\ttfamily,language=c++,numbers=left]
#include <boost/python.hpp>
char const* greet() {
    return "hello, world";
}
BOOST_PYTHON_MODULE(hello_ext) {
    using namespace boost::python;
    def("greet", greet);
}
\end{lstlisting}
\begin{itemize}
\item \verb+#include <boost/python.hpp>+: turn on Boost.Python.
\item \verb+BOOST_PYTHON_MODULE+: make a Python module.
\item \verb+def("greet", greet)+: wrap a function.
\end{itemize}
Done.
\end{frame}

\begin{frame}{
%
Boost.Python Thinks from Bottom up
%
}
\begin{itemize}
\item It is a C++ library (part of \href{http://boost.org/}{\alert{boost}}).
It operates around a fundamental C++ code base.
\item To use the comprehensive wrapping capabilities, one has to write the code
himself.
\begin{itemize}
  \item No code generation other than C++ template.
  \item C++ only and only C++.
\end{itemize}
\end{itemize}
\end{frame}

\begin{frame}[fragile]{
%
Exceptions
%
}
\begin{lstlisting}[basicstyle=\scriptsize\ttfamily,language=c++,numbers=left]
#include <boost/python.hpp>
#include <boost/python/errors.hpp>
void translate(const SomeException &err) {
    PyErr(SetString(PyExc_MemoryError, err.what());
}
BOOST_PYTHON_MODULE(error_ext) {
    using namespace boost::python;
    register_exception_translator<SomeException>(translate);
}
\end{lstlisting}
\begin{itemize}
\item \verb+#include <boost/python/errors.hpp>+
\item \verb+translate()+
\item \verb+register_exception_translator+
\end{itemize}
\end{frame}

\begin{frame}[fragile]{
%
Iterators
%
}
\begin{lstlisting}[basicstyle=\scriptsize\ttfamily,language=c++,numbers=left]
#include <boost/python.hpp>
#include <list>
struct Iterable {
  typedef std::list<int>::iterator iterator;
  iterator begin() { m_data.begin(); }
  iterator end() { m_data.end(); }
  std::list<int> m_data;
}
BOOST_PYTHON_MODULE(iter_ext) {
  using namespace boost::python;
  class_<Iterable>("Iterable", init<>))
  .add_property("iter", range(Iterable::begin, Iterable::end))
  .def("__iter__", range(Iterable::begin, Iterable::end))
  ;
}
\end{lstlisting}
\begin{itemize} \small
\item \verb+boost::python::range+: iterate from begin to end.
\item Alternately, \verb+__getitem__+ and \verb+__len__+ can help.
\end{itemize}
\end{frame}

\begin{frame}[fragile]{
%
Bi-Directional Conversions
%
}
\begin{itemize}
\item Python has GC while C++ doesn't: conversions of objects usually need to
construct intermediate objects.
\begin{itemize}
  \item It's more complex when a container is involved.
\end{itemize}
\item What to look for:
\begin{itemize}
  \item \verb+to_python_converter+: take C++ object and expose it to Python.
  \item \verb+boost::python::converter::registry+: take Python object and give
  it to C++.
\end{itemize}
\item When exchanging containers or large objects, using
\verb+boost::shared_ptr+ as a holding type helps.
\end{itemize}
\end{frame}

\begin{frame}[fragile]{
%
Obvious Caveats
%
}
\begin{itemize}
\item We need basic understanding about the CPython memory management.
\begin{itemize}
  \item Ownership of references for reference counting: {\scriptsize
  \url{http://docs.python.org/2/c-api/intro.html#reference-count-details}}
\end{itemize}
\item Be very careful about passing objects across the interpreter barrier.
\begin{itemize}
  \item Do not return intermediates, although Boost.Python checks for invalid
  returns.
  \item Some return policy copies objects.  It can hit the performance for
  large objects, or give wrong results for containers.
\end{itemize}
\end{itemize}
\end{frame}

\section{
%%%
Cython
%%%
}

\begin{frame}[fragile]{
%
What Is Cython?
%
}
\begin{itemize}
\item Cython generates C code from Python.
\begin{itemize}
  \item We need to use its command-line (\verb+cython+) like a compiler.
\end{itemize}
\item The generated code contains Python C API calls.
\begin{itemize}
  \item Can interface to both the existing \alert{C} and \alert{C++} code.
\end{itemize}
\item Some constructs can be translated into native C, and get utmost speed-up.
\item \url{http://www.cython.org/}.
\end{itemize}
\end{frame}

\begin{frame}[fragile]{
%
Cython is a Language
%
}
\begin{itemize}
\item Cython extends Python.  It is a \alert{superset} of Python.
\begin{itemize}
\item When compiling {\color{red}\verb+.py+} files, of course only Python
syntax can be used.
\end{itemize}
\item If Cython-specific syntax is used, the source code can no longer be run
by a vanilla Python VM.
\begin{itemize}
\item It should be saved as a {\color{red}\verb+.pyx+} file and compiled by
Cython compiler to binary.
\end{itemize}
\item The Cython add-on helps to generate faster code and adapt to various
libraries.
\item It's a top-down tool, in contrast to Boost.Python.
\end{itemize}
\end{frame}

\begin{frame}[fragile]{
%
Call C from Cython/Python
%
}
Let's say we have a C function \verb+caction()+:
\lstinputlisting[basicstyle=\scriptsize\ttfamily,language=C,numbers=left]{example2.c}
\end{frame}

\begin{frame}[fragile]{
%
Cython Let You Call C
%
}
\lstinputlisting[basicstyle=\scriptsize\ttfamily,language=Python,numbers=left]{example2c.pyx}
\begin{itemize}
\item \verb+cdef extern+: tells Cython that we have a C function to be used.
\item Then we just use it.  Cython takes care of the Python C API boilerplates.
\end{itemize}
\end{frame}

\begin{frame}[fragile]{
%
Useful Syntax for Wrapping
%
}
\begin{itemize}
\item \verb+cdef extern from "header.h":+
\begin{itemize}
  \item Allows Cython to check the declarations from outside headers.
\end{itemize}
\item \verb+cdef public:+
\begin{itemize}
  \item Allows Cython to generates C header files for inclusion in other C
  code.
\end{itemize}
\end{itemize}
\end{frame}

\subsection{
%%
High-Performance Cython
%%
}

\begin{frame}[fragile]{
%
Cython Is for Number-Crunching
%
}
Consider the Python version "action":
\lstinputlisting[basicstyle=\tiny\ttfamily,language=Python,numbers=left]{example2d.pyx}
\end{frame}

\begin{frame}[fragile]{
%
Let's See the Speed-Up
%
}
\begin{itemize}
\item C version runs 51.4 milliseconds; Python version runs 5.74 seconds.
\item 5.74 sec / 51.4 msec = 106 times faster.
\item If our code is for number-crunching (calculation takes the majority of
time), Python + Cython will deliver C-like performance.
\end{itemize}
\end{frame}

\begin{frame}[fragile]{
%
Technique \#1: Static Typing
%
}
\begin{itemize}
\item Just prefix ``{\color{red}\verb+cdef int+}'' to our counter:
\begin{lstlisting}[basicstyle=\scriptsize\ttfamily,language=Python,escapechar=!]
def action():
    !\colorbox{red}{cdef int}! it = 0
    while it < 1000000:
        it += 1
    print it
\end{lstlisting}
\item Then 16 times faster. (3.01 msec vs 49 msec)
\begin{lstlisting}[language=bash]
> python -m timeit -s 'from example1d import action' 'action()'
100 loops, best of 3: 3.01 msec per loop
\end{lstlisting}
\item Because Cython can use the type information to generate native C code
instead of C API calls.
\end{itemize}
\end{frame}

\begin{frame}[fragile]{
%
Technique \#2: Direct Indexing
%
}
Provide array dimensions for direct indexing:
\begin{lstlisting}[basicstyle=\tiny\ttfamily,language=Python,numbers=left,escapechar=!]
import numpy as np
cimport numpy as cnp
def action():
    !\colorbox{red}{cdef cnp.ndarray[cnp.double\_t, ndim=2]}! arr0 = np.empty([1000,1000], dtype='float64')
    arr0.fill(0)
    !\colorbox{red}{cdef cnp.ndarray[cnp.double\_t, ndim=2]}! arr1 = np.empty([1000,1000], dtype='float64')
    arr1.fill(1)
    cdef int it = 1
    cdef int jt
    while it < 999:
        jt = 1
        while jt < 999:
            arr0[it, jt] += arr1[it-1, jt-1]
            arr0[it, jt] += arr1[it-1, jt  ]
            arr0[it, jt] += arr1[it-1, jt+1]
            arr0[it, jt] += arr1[it  , jt+1]
            arr0[it, jt] += arr1[it+1, jt+1]
            arr0[it, jt] += arr1[it+1, jt  ]
            arr0[it, jt] += arr1[it+1, jt-1]
            arr0[it, jt] += arr1[it  , jt-1]
            jt += 1
        it += 1
    assert 7968032 == arr0.sum()
\end{lstlisting}
\end{frame}

\begin{frame}[fragile]{
%
Direct Indexing Is Way Faster
%
}
\begin{itemize}
\item Use NumPy API for indexing: 5.74 sec.
\begin{lstlisting}[language=bash]
> python -m timeit -s 'from example2d import action' 'action()'
10 loops, best of 3: 5.74 sec per loop
\end{lstlisting}
\item Direct indexing: 139 msec. 
\begin{lstlisting}[language=bash]
> python -m timeit -s 'from example2a import action' 'action()'
10 loops, best of 3: 139 msec per loop
\end{lstlisting}
\item Static typing and direct indexing make the code run \alert{41} times
faster.
\item It's only 139 msec / 51.4 msec = 2.7 times slower than the C version.
\begin{itemize}
  \item Good enough for many uses.
\end{itemize}
\end{itemize}
\end{frame}

\section{
%%%
Conclusions
%%%
}

\begin{frame}{
%
Choose Your Wrapping Tool
%
}
\begin{itemize}
\item Python is two orders of magnitude slower than C.
\item If you want to wrap C++, choose Boost.Python.
\begin{itemize}
  \item It thinks around C++.  The wrapper should be built from bottom up.
\end{itemize}
\item If you want to wrap C, choose Cython.
\begin{itemize}
  \item It thinks around Python.  The wrapper should be built from top down.
\end{itemize}
\item When working with NumPy arrays, static typing along with direct indexing
makes the number-crunching very fast, just 2 times slower than the C counter
part.
\end{itemize}
\end{frame}

\end{document}

% vim: set spell:
