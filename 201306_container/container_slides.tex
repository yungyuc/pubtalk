\documentclass[dvips,xcolor=pst,14pt]{beamer}
%\documentclass[dvips,handout,xcolor=pst]{beamer}
%\documentclass[letterpaper]{article}
%\usepackage{beamerarticle}
%\newcommand{\newblock}{} % a hack for bibtex.
\usepackage{color}
\usepackage{pstricks}
%\usepackage{hyperref}
%\usepackage[authoryear]{natbib}
\usepackage{graphicx}
\usepackage{multimedia}
\usepackage{pgfpages}
\usepackage{arydshln}
\usepackage{commath}
\usepackage{vector}
%\usepackage{cancel}
\usepackage{ulem}
\usepackage{listings}
%\usepackage{times}

\graphicspath{{pysci_eps/}}

\newcommand{\defeq}{\ensuremath{\buildrel {\text{def}}\over{=}}} 

\usetheme{CambridgeUS}
%\usefonttheme[stillsansseriftext]{serif}
\usefonttheme[onlymath]{serif}

%\setbeameroption{show notes}
%\setbeameroption{show only notes}
%\pgfpagesuselayout{2 on 1}[letterpaper,border shrink=0.2in]

\lstset{basicstyle=\footnotesize\ttfamily,%
keywordstyle=\color{blue},%
stringstyle=\color{magenta},%
commentstyle=\color{red},%
breaklines=true}
%\lstset{framextopmargin=50pt,frame=bottomline}

\title[Containers]{The Usefulness of Python: Containers}
%
\author[\href{http://solvcon.net/yyc/}{Yung-Yu Chen}]%
{\href{http://solvcon.net/yyc/}{Yung-Yu Chen} \\ {\scriptsize
\url{yyc@solvcon.net}}}
%
\institute[PyHUG]{Python Hsinchu User Group}
%
\date[2013/6/3]{June 3, 2013}

\begin{document}

\begin{frame}
\titlepage
\end{frame}

% \begin{frame}{
% %
% Our Topics Today
% %
% }
% \tableofcontents
% \end{frame}

\section{
%%%
Program for Productivity
%%%
}

\subsection{
%%
Take List for Example
%%
}

\begin{frame}[fragile]{
%
What's a Container
%
}
\begin{itemize}
\item OK, here's some C++ code for an \verb+int+ list:
  \lstinputlisting[%
    basicstyle=\scriptsize\ttfamily,numbers=left,language=C++]{list.cpp}
\end{itemize}

\begin{flushleft}\tiny Code is taken from
\url{http://www.csci.csusb.edu/dick/samples/stl.html#Lists}.
\end{flushleft}
\end{frame}

\begin{frame}[fragile]{
%
What Does It Take for a List?
%
}
\begin{itemize}
\item Many many things.
\item A very simple (and primitive) implementation takes 157 lines of C code.
\item The first 10 lines:
  \lstinputlisting[%
    basicstyle=\scriptsize\ttfamily,numbers=left,%
    firstline=1,lastline=10,language=C]{list.c}
\item And it's also a list for \verb+int+ only!
\end{itemize}

\begin{flushleft}\tiny Code is taken from
\url{http://www.thegeekstuff.com/2012/08/c-linked-list-example/}.
\end{flushleft}
\end{frame}

\begin{frame}[fragile]{
%
Let's See a Python List
%
}
\begin{itemize}
\item 2 (or 1?) lines of code:
  \lstinputlisting[%
    basicstyle=\scriptsize\ttfamily,numbers=left,language=Python]{list.py}
  \begin{itemize}
  \item This slide becomes a bit too blank.
  \end{itemize}
\item It's so good to have Python.
\end{itemize}
\end{frame}

\subsection{
%%
Python Containers
%%
}

\begin{frame}[fragile]{
%
What Are Python Containers?
%
}
\begin{itemize}
\item In fact there is no such a type called ``container'' in Python.
  \begin{itemize}
  \item It's just a common name to refer objects that can hold other objects.
  \item Like \verb+list+, \verb+stack+, \verb+vector+, \verb+map+, etc., in
  C++.
  \item The dynamicity makes containers in Python more powerful than in C++.
  \end{itemize}
\item Three categories of Python containers:
\begin{itemize}
  \item \href{http://docs.python.org/2/library/stdtypes.html#sequence-types-str-unicode-list-tuple-bytearray-buffer-xrange}{%
    Sequence types (\texttt{list}, \texttt{tuple}, etc.)}
  \item \href{http://docs.python.org/2/library/stdtypes.html#set-types-set-frozenset}{%
    Set types (\texttt{set} and \texttt{fronzeset}).}
  \item \href{http://docs.python.org/2/library/stdtypes.html#mapping-types-dict}{%
    Mapping types (\texttt{dict}).}
\end{itemize}
\item They fit your brains.
\end{itemize}
\end{frame}

\section{
%%%
Sequence
%%%
}

\subsection{
%%
List
%%
}

\begin{frame}[fragile]{
%
\texttt{list}: A General Sequence Type
%
}
\begin{itemize}
\item \texttt{str} might be the most used sequence type, but it is for
character strings only.
  \begin{itemize}
  \item \texttt{list} can hold anything.
  \end{itemize}
\item \texttt{[]} or \texttt{list()} constructs a list for you:
  \begin{lstlisting}[basicstyle=\scriptsize\ttfamily,numbers=left,language=Python]
>>> la = []
>>> lb = list()
>>> print(la, lb)
([], [])
  \end{lstlisting}
\item Some built-ins return a list:
  \begin{lstlisting}[basicstyle=\scriptsize\ttfamily,numbers=left,language=Python]
>>> a = range(10)
>>> print(type(a), a)
(<type 'list'>, [0, 1, 2, 3, 4, 5, 6, 7, 8, 9])
  \end{lstlisting}
\end{itemize}
\end{frame}

\begin{frame}[fragile]{
%
Sort a \texttt{list}
%
}
\begin{itemize}
\item Simple sort:
  \begin{lstlisting}[basicstyle=\scriptsize\ttfamily,numbers=left,language=Python]
>>> a = [87, 82, 38, 56, 84]
>>> b = sorted(a) # b is a new list.
>>> print(b)
[38, 56, 82, 84, 87]
>>> a.sort() # this method does in-place sort.
>>> print(a)
[38, 56, 82, 84, 87]
  \end{lstlisting}
\item Not-so-simple sort:
  \begin{lstlisting}[basicstyle=\scriptsize\ttfamily,numbers=left,language=Python]
>>> a = [('a', 0), ('b', 2), ('c', 1)]
>>> print(sorted(a)) # sorted with the first value.
[('a', 0), ('b', 2), ('c', 1)]
>>> print(sorted(a, key=lambda k: k[1])) # use the second.
[('a', 0), ('c', 1), ('b', 2)]
  \end{lstlisting}
\end{itemize}
\end{frame}

\begin{frame}[fragile]{
%
List Comprehension
%
}
\begin{itemize}
\item List comprehension is a very useful technique to construct a list from
another iterable:
  \begin{lstlisting}[basicstyle=\scriptsize\ttfamily,numbers=left,language=Python]
>>> values = [10.0, 20.0, 30.0, 15.0]
>>> print([it/10 for it in values])
[1.0, 2.0, 3.0, 1.5]
  \end{lstlisting}
\item List comprehension can even be nested:
  \begin{lstlisting}[basicstyle=\scriptsize\ttfamily,numbers=left,language=Python]
>>> values = [[10.0, 1.0], [20.0, 2.0], [30.0, 3.0], [15.0, 1.5]]
>>> print([jt for it in values for jt in it])
[10.0, 1.0, 20.0, 2.0, 30.0, 3.0, 15.0, 1.5]
  \end{lstlisting}
  But usually nested list comprehension is not easy to understand, and not a
  good idea.
\end{itemize}
\end{frame}

\begin{frame}[fragile]{
%
Recipe: Calculation
%
}
\begin{itemize}
\item Built-in calculation functions for iterables:
  \begin{lstlisting}[basicstyle=\scriptsize\ttfamily,numbers=left,language=Python]
>>> values = [10.0, 20.0, 30.0, 15.0]
>>> min(values), max(it for it in values)
(10.0, 30.0)
>>> sum(values)
75.0
>>> sum(values)/len(values)
18.75
  \end{lstlisting}
\end{itemize}
\end{frame}

\subsection{
%%
Tuple
%%
}

\begin{frame}[fragile]{
%
\texttt{tuple}: Immutable Sequence
%
}
\begin{itemize}
\item A \texttt{tuple} can also hold anything, but cannot be changed once
constructed.
\item It can be created with \texttt{()} or \texttt{tuple()}:
  \begin{lstlisting}[basicstyle=\scriptsize\ttfamily,numbers=left,language=Python]
>>> ta = (1)
>>> print(type(ta), ta)
(<type 'int'>, 1)
>>> ta = (1,)
>>> print(type(ta), ta)
(<type 'tuple'>, (1,))
>>> ta[0] = 2
Traceback (most recent call last):
  File "<stdin>", line 1, in <module>
TypeError: 'tuple' object does not support item assignment
  \end{lstlisting}
\end{itemize}
\end{frame}

\section{
%%%
Set
%%%
}

\begin{frame}[fragile]{
%
\texttt{set}: Distinct Elements
%
}
\begin{itemize}
\item A \texttt{set} holds any hashable element, and its elements are distinct:
  \begin{lstlisting}[basicstyle=\scriptsize\ttfamily,numbers=left,language=Python]
>>> sa = {1, 2, 3}
>>> print(type(sa), sa)
(<type 'set'>, set([1, 2, 3]))
>>> print({1, 2, 2, 3}) # no duplication is possible.
set([1, 2, 3])
>>> len({1, 2, 2, 3})
3
  \end{lstlisting}
\item It's unordered:
  \begin{lstlisting}[basicstyle=\scriptsize\ttfamily,numbers=left,language=Python]
>>> [it for it in {3, 2, 1}]
[1, 2, 3]
>>> [it for it in {3, 'q', 1}]
['q', 1, 3]
>>> 'q' < 1
False
  \end{lstlisting}
\end{itemize}
\end{frame}

\begin{frame}[fragile]{
%
Add/Remove Elements
%
}
\begin{itemize}
\item Add elements after construction of the set:
  \begin{lstlisting}[basicstyle=\scriptsize\ttfamily,numbers=left,language=Python]
>>> sa = {1, 2, 3}
>>> sa.add(1)
>>> sa
set([1, 2, 3])
>>> sa.add(10)
>>> sa
set([1, 2, 3, 10])
  \end{lstlisting}
\item Remove elements:
  \begin{lstlisting}[basicstyle=\scriptsize\ttfamily,numbers=left,language=Python]
>>> sa = {1, 2, 3, 10}
>>> sa.remove(5) # err with non-existing element
Traceback (most recent call last):
  File "<stdin>", line 1, in <module>
KeyError: 5
>>> sa.discard(2) # really discard an element
>>> sa
set([1, 3, 10])
  \end{lstlisting}
\end{itemize}
\end{frame}

\begin{frame}[fragile]{
%
Set Operations
%
}
\begin{itemize}
\item Subset or superset:
  \begin{lstlisting}[basicstyle=\scriptsize\ttfamily,numbers=left,language=Python]
>>> {1, 2, 3} < {2, 3, 4, 5} # subset
False
>>> {2, 3} < {2, 3, 4, 5}
True
>>> {2, 3, 4, 5} > {2, 3} # superset
True
  \end{lstlisting}
\item Union and intersection:
  \begin{lstlisting}[basicstyle=\scriptsize\ttfamily,numbers=left,language=Python]
>>> {1, 2, 3} | {2, 3, 4, 5} # union
set([1, 2, 3, 4, 5])
>>> {1, 2, 3} & {2, 3, 4, 5} # intersection
set([2, 3])
>>> {1, 2, 3} - {2, 3, 4, 5} # difference
set([1])
  \end{lstlisting}
\end{itemize}
\end{frame}

\begin{frame}[fragile]{
%
Recipe: Count Unique Elements
%
}
\begin{itemize}
\item A set can be used with a sequence to quickly calculate unique elements:
  \begin{lstlisting}[basicstyle=\scriptsize\ttfamily,numbers=left,language=Python]
>>> data = [1, 2.0, 0, 'b', 1, 2.0, 3.2]
>>> sorted(set(data))
[0, 1, 2.0, 3.2, 'b']
  \end{lstlisting}
\item But there's a problem: It doesn't support unhashable objects:
  \begin{lstlisting}[basicstyle=\scriptsize\ttfamily,numbers=left,language=Python]
>>> data = [dict(a=200), 1, 2.0, 0, 'b', 1, 2.0, 3.2]
>>> set(data)
Traceback (most recent call last):
  File "<stdin>", line 1, in <module>
TypeError: unhashable type: 'dict'
  \end{lstlisting}
  Read the Python Cookbook for a solution :-)
\end{itemize}
\end{frame}

\section{
%%%
Mapping
%%%
}

\begin{frame}[fragile]{
%
\texttt{dict}: The Mapping Type
%
}
\begin{itemize}
\item A \texttt{dict} stores any number of key-value pairs.  It is the most used
Python container since it's everywhere for Python namespace.
  \begin{lstlisting}[basicstyle=\scriptsize\ttfamily,numbers=left,language=Python]
>>> {'a': 10, 'b': 20} == dict(a=10, b=20)
True
>>> da = {1: 10, 2: 20} # any hashable can be a key
>>> da[1] + da[2]
30
>>> class SomeClass(object):
...     pass
... 
>>> print(type(SomeClass().__dict__))
<type 'dict'>
  \end{lstlisting}
\end{itemize}
\end{frame}

\begin{frame}[fragile]{
%
Access Contents
%
}
\begin{itemize}
\item To test whether something is in a dictionary or not:
  \begin{lstlisting}[basicstyle=\scriptsize\ttfamily,numbers=left,language=Python]
>>> da = {1: 10, 2: 20}
>>> 3 in da
False
  \end{lstlisting}
\item Access a key-value pair:
  \begin{lstlisting}[basicstyle=\scriptsize\ttfamily,numbers=left,language=Python]
>>> da[3] # it fails for 3 is not in the dictionary
Traceback (most recent call last):
  File "<stdin>", line 1, in <module>
KeyError: 3
>>> print(da[3] if 3 in da else 30) # works but wordy
30
>>> da.get(3, 30) # it's the way to go
30
>>> da # indeed we don't have 3 as a key
{1: 10, 2: 20}
>>> da.setdefault(3, 30) # how about this?
30
>>> da # we added 3 into the dictionary!
{1: 10, 2: 20, 3: 30}
  \end{lstlisting}
\end{itemize}
\end{frame}

\begin{frame}[fragile]{
%
Iterating a \texttt{dict}
%
}
\begin{itemize}
\item Iterating a \texttt{dict} automatically gives you its keys:
  \begin{lstlisting}[basicstyle=\scriptsize\ttfamily,numbers=left,language=Python]
>>> da = {1: 10, 2: 20}
>>> ','.join('%s'%key for key in da)
'1,2'
>>> ','.join('%d'%da[key] for key in da)
'10,20'
  \end{lstlisting}
\item \texttt{items()} and \texttt{iteritems()} give you both key and value at
once:
  \begin{lstlisting}[basicstyle=\scriptsize\ttfamily,numbers=left,language=Python]
>>> da.items() # returns a list
[(1, 10), (2, 20)]
>>> da.iteritems() # returns an iterator
<dictionary-itemiterator object at 0xa35050>
>>> ','.join('%s:%s'%(key, value) for key, value in da.iteritems())
'1:10,2:20'
  \end{lstlisting}
\end{itemize}
\end{frame}

\begin{frame}[fragile]{
%
Dictionary Views
%
}
\begin{itemize}
\item A dictionary view changes with the dictionary:
  \begin{lstlisting}[basicstyle=\scriptsize\ttfamily,numbers=left,language=Python]
>>> da = {1: 10, 2: 20}
>>> daiit = da.iteritems() # an iterator
>>> daiit
<dictionary-itemiterator object at 0xa350a8>
>>> davit = da.viewitems() # a view object
>>> davit
dict_items([(1, 10), (2, 20)])
>>> da[3] = 30 # change the dictionary
>>> ','.join('%s:%s'%(key, value) for key, value in daiit)
Traceback (most recent call last):
  File "<stdin>", line 1, in <module>
  File "<stdin>", line 1, in <genexpr>
RuntimeError: dictionary changed size during iteration
>>> ','.join('%s:%s'%(key, value) for key, value in davit)
'1:10,2:20,3:30'
  \end{lstlisting}
\end{itemize}
\end{frame}

\section{
%%%
\texttt{collections}
%%%
}

\begin{frame}[fragile]{
%
Collections Abstract Base Classes
%
}
\begin{itemize}
\item OK, what I really wanted to talked about in this talk was the
\href{http://docs.python.org/2/library/collections.html#collections-abstract-base-classes}{collections
ABCs}.
  \begin{itemize}
  \item But it seemed impossible to prepare a talk about them in 3 hours.
  \item Sorry.
  \item I stop here.
  \end{itemize}
\item When you want to make your own containers (data structures), they are
extremely handy.
  \begin{itemize}
  \item For example, an
  \href{http://code.activestate.com/recipes/576694/}{OrderedSet} that uses
  \href{http://docs.python.org/2/library/collections.html#collections.MutableSet}{\texttt{collections.MutableSet}}.
  \end{itemize}
\item Maybe next time.
\end{itemize}
\end{frame}

\section*{
%%%
%%%
}

\begin{frame}{
%
%
}
\begin{center}
Thanks!
\end{center}
\end{frame}

\end{document}

% vim: set spell:
